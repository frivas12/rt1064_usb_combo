\section{Preprocessors}\label{sec:preprocessors}
In addition to the reserved tags, the XML preprocessor also supports macros for importing
    information handled by the preprocessor.
Macros apply to all attributes in an XML document.
Macros are currently simple; attributes with contents starting with the character '\$' are
    sent through the macro preprocessor.
To literally declare a '\$' character at the start of an attribute use ``\$\$''.

\subsection*{\$PATH:}\label{sec:preprocessors/path}
The path macro allows for path-relative addressing in attributes.
This uses the same system as the ``Import'' and ``Include'' tags.

Commonly, this is used in compiled/processed XMLs, such as config LUT entries.

For example, a config LUT XML will usually be compiled into a binary file in the ``bin/''
    subdirectory.
To do this, the ``Struct'' tag will use the ``binary'' attribute with the path macro.
The example below shows a structure that will be compiled into the path 
    ``bin/out.bin'' relative to the XML's location.
\begin{lstlisting}
<?xml version="1.0">
<Struct binary="$PATH:bin/out.bin">
    <Signature>
        <StructID value="4" />
        <ConfigID value="0xFFF0" />
    </Signature>
    <Data>
        <Flags value="0" />
    </Data>
</Struct>
\end{lstlisting}