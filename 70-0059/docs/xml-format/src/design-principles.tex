\section{Design Principles}\label{sec:design-principles}
In the previous XML format, xml files were declared as APT commands that were sent
    to the controller.
As a result, XML files declared the structure and parameters for the command in
    byte form, making the files difficult to understand without referencing
    the APT documentation and/or the firmware's source code.

The new XML format seeks to separate how the XMLs are written and the APT commands
    sent to the controller.
In the new format, XMLs are easier to understand and easier to maintain.
In addition, the new XML format provides a cohesive structure for XML files
    for configuration of the controller.
A large middleware layer in form of the SystemXMLParser has arisen to convert the
    structure of each XML file into a configuration to send to the controller.

As the XMLs stored define the parameters for configuring a controller and/or one-wire
    device, \textbf{extensive} XMLs act as the core pillar of every XML document.
Extensiveness requires that one XML should hold all the information needed to
    configure a device.
This makes configuration a simple as selecting a file.
Thus, only configuration XMLs are the only XMLs loaded by other programs.


If extensiveness was the only pillar, however, the modular nature of the MCM controllers would
    make this design principle unfeasible.
The pillar of \textbf{indirection} bridges this issue, allowing for extensive files to be split
    into separate files.
This pillar creates a system where XMLs can include elements from other XMLs, allowing for a single file
    to branch into multiple files in the directory.
Furthermore, extensive XMLs can then reuse the same file, reducing file size and automatically
    propagating changes.
