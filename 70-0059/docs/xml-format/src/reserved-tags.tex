\section{Reserved Tags}\label{sec:reserved-tags}
Certain tag names are reserved as special tags.
As a result, these tag names are \textbf{must} be defined for their use as defined in this section.

\subsection*{Import}\label{sec:reserved-tags/import}
The ``Import'' tag allows for XML elements from other files to be included where the import tag is declared.
To use this tag, declare the ``href'' and ``xpath'' attributes.
The ``href'' attribute declares a relative path to the XML file to reference.
The ``xpath'' attribute declares an XPath path in the XML to the element to be imported.

For example, assume that file ``B.xml'' exists in the subdirectory ``src'' with the following
    contents:
\begin{lstlisting}
<?xml version="1.0" ?>

<B>
    <MyTag>
        <MyElement myAttr="C" />
    </MyTag>
    <SomeoneElse otherAttr="D" />
</B>
\end{lstlisting}

If ``A.xml'' wants to import the MyTag block, then one would use the ``Import'' tag as follows.
\begin{lstlisting}
<Import href="src/B.xml" xpath="/B/MyTag" />
\end{lstlisting}

After ``A.xml'' is preprocessed, then the ``Import'' tag will be replaced with the ``MyTag''
    element, including all of that tag's content.

\subsection*{Include}\label{sec:reserved-tags/include}
The ``Include'' tag acts as a useful shorthand for a common use-case of the ``Import'' tag--full
    content importing.
Functionally, it is equivalent to\dots\\
\lstinline{<Import href="your/href/here" xpath="/" />}

To use this tag, declare the ``href'' attribute.
Just like the ``Import'' tag, the ``href'' attribute declares a relative path to an XML file to
    reference.