\section{Utilities}\label{sec:utilities}

In the ``Misc'' folder of the repository, the ``lut-xml-templates'' folder contains tools XML creation.
All of the tools are written in python 3.

\subsection*{converter.py}
The converter tool allows for the previous XML version to be converted into the new XML version.
The first command-line argument to the tool is the old XML to convert.
The second (optional) command-line argument defines what path the converted XML should be.
If none is provided, the working directory will be used to output an XML with the same name as the input XML.
\textbf{Notice:}  Do not use the default output path when the input file is in the same directory.

Currently, joysticks and steppers are supported.
Joysticks convert directly into the new XML format without the need for editing.
These can be directly placed into the ``xml/system/joysticks/'' directory.
Steppers convert into a ``StageParameters'' XML--the XMLs stored in the ``xml/system/device/full/stepper/''
    directory.

\subsection*{splitter.py}
As the Plug-and-Play system splits configurations into multiple structures, porting device parameters
    requires the splitter tool to separate the structures.
The first command line argument is the path to the full configuration parameters for
    a device (e.g. ``StageParameters'').
The second (optional) argument defines the path to the output directory for the generated XML files.
If no argument is given, then the directory will share the file name of the passed XML.

In the output directory, files will be generated for each structure in addition to a device file.
Each struct file will have the name of the original XML file followed by and underscore and the name of the
    struct.
For the device file, the name will have an ``_Device'' appended to the original name.
Unlike the converter tool, the splitter tool requires\dots
\begin{enumerate}
    \item Setting the config id for each generated configuration structure.
    \item Setting the device configuration set.
    \item Setting the device signature.
\end{enumerate}

For Plug-and-Play devices, converted XMLs from ``converter.py'' will then be passed through this
    tool to separate their structures.

\subsection*{searcher.py}
To aid in sharing common structures, the searcher tool finds is any struct XMLs in a directory matches in
    content with a struct encoded in the config struct database.
The first argument is the path to the MS Access database containing the configuration structures.
For the remaining arguments (at least one), these will represent path(s) to XMLs to compare with the
    config database.

This tool can be used in conjunction with the splitter tool to see if any generated configs
    match an already existing configuration.

\subsection*{make_onewire.py}
With the make-onewire tool, developers can easily create one-wire XMLs for a given device.
To make the one-wire XML, enter the ``--slot-card'' and ``--device-id'' options.
These will define the device signature of the device.
Optionally, the ``--part-number'' can also be passed.
Then, pass the ``--template'' file, an XML that tells the program how to order the incoming files.
Optionally, the ``-output'' XML can be defined.
Furthermore, the ``--delete'' option allows users to remove all the used files if an output was generated.
Finally, pass in individual XML files into the program.
This will pack them into a single one-wire XML.

Notice:  the one-wire tool does not know if the total XML will be able to be stored in the EEPROM chip.


\subsection*{make_config_set.py}
Using the previous tools, the make-config-set tool will populate the fields in the device XML using the other
    XMLs in the directory.

\subsection*{joiner.py}
Finally, the joiner tool can take XMLs (and the config database) to recreate the full device parameter XML
    like the one generated from the converter tool.
To do this, the joiner tool requires a device's configuration set and \emph{all} configurations in that set.
Configurations, however, can be imported from both struct XMLs and one-wire XMLs.
To use the tool, provide all of the input files as command line arguments.
Optionally, the ``--output'' option allows users to set the path of the output file.
Otherwise, the output file will be ``out.xml''.